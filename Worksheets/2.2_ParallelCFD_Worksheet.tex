\documentclass{article}

\usepackage[margin=2cm]{geometry}
\usepackage[most]{tcolorbox}

\newtcolorbox{smalltextbox}[1][]{%
  sharp corners,
  enhanced,
  colback=white,
  height=4cm,
  attach title to upper,
  #1
}

\newtcolorbox{largetextbox}[1][]{%
  sharp corners,
  enhanced,
  colback=white,
  height=8cm,
  attach title to upper,
  #1
}

\begin{document}

\section*{2.2. Parallel CFD Simulations -- Worksheet}

\begin{smalltextbox}[fontupper=\bfseries\large]

What are the categories to distinguish parallel programming models?

\end{smalltextbox}


\begin{smalltextbox}[fontupper=\bfseries\large]%[colupper=blue,fontupper=\bfseries\large]

What parallel programming model is typically used for CFD and why?

\end{smalltextbox}


\begin{smalltextbox}[fontupper=\bfseries\large]

Which main incredients need to be devised for parallel CFD?

\end{smalltextbox}


\begin{largetextbox}[fontupper=\bfseries\large]

What parallel communication pattern results from the finite volume matrix assembly? Exemplify for the advection term.

\end{largetextbox}

\pagebreak[4]

\begin{smalltextbox}[fontupper=\bfseries\large]

List the de-facto standard library deployed in parallel CFD. Are there alternatives?

\end{smalltextbox}

\begin{largetextbox}[fontupper=\bfseries\large]

Your questions:

\end{largetextbox}

\end{document}
